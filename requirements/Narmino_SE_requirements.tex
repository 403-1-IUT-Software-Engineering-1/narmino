% !TEX TS-program = xelatex

\documentclass[12pt]{article}
% !TEX TS-program = xelatex
% !TeX root=Narmino_SE_Project_Proposal.tex


\usepackage[utf8]{inputenc}
\usepackage{amsmath,amsfonts,amssymb,amsthm}
\usepackage{mathtools}

\usepackage{geometry}
\geometry{a4paper,hmargin=20mm,vmargin=25mm,head=3ex,foot=8ex}
\usepackage{titlesec} %[newparttoc]
\usepackage{fancyhdr}
\usepackage{graphicx}
\usepackage[usenames,dvipsnames]{color,xcolor}
\usepackage{listings}
\usepackage{fourier}
\usepackage[most]{tcolorbox}
\usepackage{xcolor}
\usepackage[colorlinks=true, linkcolor=Blue,urlcolor=Blue,citecolor=Blue, pagebackref, linkbordercolor=Blue]{hyperref} %%%
\usepackage{xurl}
\usepackage{enumitem}
\usepackage{float}
%\usepackage{subcaption}
\usepackage[font=small,skip=0pt]{caption}
\usepackage{subfigure}
%%%%%%%%%%%%%%%%%%%%%%%
\usepackage{multirow,array}

\usepackage{algorithm}
\usepackage{algorithmic}
\usepackage{realboxes}

\usepackage{tikz}
\usetikzlibrary{mindmap}

\usepackage{xpatch}
\xpretocmd\headrule{\color{Blue}}{}{\PatchFailed}
%\newlength{\myheight}
%\lfoot{}
%\settoheight{\myheight}{\thepage}
%\cfoot{\color{Blue}\thepage}
\renewcommand\headrulewidth{3pt}
%\renewcommand\footrulewidth{0pt}



%%%%%%%%%%%%%%%%%%%%%%%%%% TOC

\usepackage{titletoc}

%%%%%%%%%%%%%%%%%%%%%%%
\usepackage[extrafootnotefeatures]{xepersian} %
\settextfont{HM FMitra}   %{HM FMitra}
\setdigitfont{HM FMitra}   %{HM FMitra}
\setlatintextfont{Times New Roman} %[Scale=.9]
\defpersianfont\titrfont{HM FMitra}   %{HM FMitra}
%\paragraphfootnotes
\usepackage{perpage}
\MakePerPage{footnote}
\linespread{1.6}



\DeclarePairedDelimiter{\abs}{\lvert}{\rvert}
%%%%%%%%%%%%%%%%%%
\setlength{\footmarkwidth}{0pt}

\setlength{\hsize}{\textwidth}
\paragraphfootnotes
%\makeatletter
%\newif\ifLTRfootnoterule
%\let\@footnotetext\m@mold@footnotetext
%\gdef\mem@makecol{%
%  \m@m@makecolintro
%  \ifvoid\footins
%  \ifvoid\footinsv@r\else \LTRfootnoteruletrue\fi
%    \setbox\@outputbox \box\@cclv
%  \else
%    \m@mopfootnote
%  \fi
%  \m@mdoextrafeet
%  \m@m@makecolfloats
%  %\m@mopsidebar
%  \m@m@makecoltext
%  \global \maxdepth \@maxdepth
%  }
%\renewcommand*{\@footstartv@r}{%
%    \vskip 3mm %\bigskipamount
%    \leftskip=\z@
%    \rightskip=\z@
%    \ifLTRfootnoterule\footnoterule\fi}
%\makeatother
%%%%%%%%%%%%%%
%%%%%%%%%%%%%%%%%
%\setcounter{secnumdepth}{4}
%\setcounter{tocdepth}{3}
\newtcolorbox{SAMBox}[3]{colback=#1, colframe=#2, fonttitle=\bfseries,title={\LTR #3},breakable,enhanced}
\newtcolorbox{SAMBoxP}[3]{colback=#1, colframe=#2, fonttitle=\bfseries,title={#3},breakable,enhanced}
%%%%%%%%%%%%%%%%%


%\lstset{ 
%	language=Python,
%  backgroundcolor=\color{red!10},   % choose the background color; you must add \usepackage{color} or \usepackage{xcolor}; should come as last argument
%  basicstyle=\fontsize{10}{10}\selectfont \tt,        % the size of the fonts that are used for the code
%  breakatwhitespace=false,         % sets if automatic breaks should only happen at whitespace
%  breaklines=true,                 % sets automatic line breaking
%  captionpos=b,                    % sets the caption-position to bottom
%  commentstyle=\color{purple!40!black},    % comment style
%  deletekeywords={...},            % if you want to delete keywords from the given language
%  escapeinside={\%*}{*)},          % if you want to add LaTeX within your code
%  extendedchars=true,              % lets you use non-ASCII characters; for 8-bits encodings only, does not work with UTF-8
%  frame=single,	                   % adds a frame around the code   %  frame=L,
%  keepspaces=true,                 % keeps spaces in text, useful for keeping indentation of code (possibly needs columns=flexible)
%  keywordstyle=\color{blue},       % keyword style
% % language=Octave,                 % the language of the code
%  morekeywords={*,...},            % if you want to add more keywords to the set
%  numbers=left,                    % where to put the line-numbers; possible values are (none, left, right)
%  numbersep=5pt,                   % how far the line-numbers are from the code
%  numberstyle=\tiny\color{green!40!black}, % the style that is used for the line-numbers
%  rulecolor=\color{black},         % if not set, the frame-color may be changed on line-breaks within not-black text (e.g. comments (green here))
%  showspaces=false,                % show spaces everywhere adding particular underscores; it overrides 'showstringspaces'
%  showstringspaces=false,          % underline spaces within strings only
%%  showtabs=false,                  % show tabs within strings adding particular underscores
%  stepnumber=1,                    % the step between two line-numbers. If it's 1, each line will be numbered
%  stringstyle=\color{orange},     % string literal style
%%  tabsize=2,	                   % sets default tabsize to 2 spaces
%%  title=\lstname,                   % show the filename of files included with \lstinputlisting; also try caption instead of title
%%  belowcaptionskip=\baselineskip,
%  xleftmargin=\parindent,
%}



\lstset{ 
	language=C++,
  	backgroundcolor=\color{red!10},  
  	basicstyle=\fontsize{10}{10}\selectfont \tt,    
 	 breakatwhitespace=false,  
  	breaklines=true,  
 	 captionpos=b,  
	  commentstyle=\color{purple!40!black},    
	  deletekeywords={...},          
	  escapeinside={\%*}{*)},  
	  extendedchars=true,  
 	 frame=single,	
 	 keepspaces=true,    
 	 keywordstyle=\color{blue},   
 	 morekeywords={*,...},        
 	 numbers=left, 
  	numbersep=5pt,     
 	 numberstyle=\tiny\color{green!40!black}, 
 	 rulecolor=\color{black},  
 	 showspaces=false, 
 	 showstringspaces=false,   
	%  showtabs=false,         
 	stepnumber=1,  
	 stringstyle=\color{orange},    
	%  tabsize=2,	 
	%  title=\lstname,  
	  %belowcaptionskip=\baselineskip,
 	 xleftmargin=0pt%\parindent,
}


\lstdefinestyle{DOS}
{
    backgroundcolor=\color{black},
    basicstyle=\scriptsize\color{white}\ttfamily
}




\newcommand{\code}[1]{\lr{\lstinline{#1}}}

%%%%%%%%%%%%%%%%%

\definecolor{Blue}{rgb}{0,0,.55}
\definecolor{Blue2}{rgb}{.55,0,.55}


\titleformat{\section}
{\color{Blue} \Large\bfseries}
{\thesection}{1em}
{}[\titlerule]

\titleformat{\subsection}
{\color{Blue} \large\bfseries}
{\thesubsection}
{1em}{}
\titleformat{\subsubsection}
{\color{Blue} \bfseries}
{\thesubsubsection}
{1em}{}



%%TITLE TOC
\titlecontents{part}[0pt]{\color{Blue}\large\bfseries\protect\addvspace{15pt}}%
{}{\color{Blue}\partname{} }%
{\color{Blue}\enspace\hfill\contentspage\enspace}%
[\titlerule]



%%%%%%%%%%%%%%%%%%%%%

\newtheorem{theorem}{قضیه}[section]
\newtheorem{lemma}[theorem]{لم}
\newtheorem{proposition}[theorem]{گزاره}
\theoremstyle{definition}
\newtheorem{definition}[theorem]{تعریف}
\newtheorem{example}[theorem]{مثال}
\newtheorem{remark}[theorem]{ملاحظه}
\newtheorem{obs}[theorem]{مشاهده}
\newtheorem{rull}[theorem]{قانون}
\newtheorem*{namad}{نمادگذاری}

%%%%%%%%%%%%%%%%%%%%

\graphicspath{{Figs/}}

%%%%%%%%%%%%%%%%%%%%%%%%
%\renewcommand\bibname{مراجع}
\makeatletter
\renewenvironment{thebibliography}[1]
     {\section{مراجع}% <-- this line was changed from \chapter* to \section*
      \@mkboth{\MakeUppercase\bibname}{\MakeUppercase\bibname}%
      \list{\@biblabel{\@arabic\c@enumiv}}%
           {\settowidth\labelwidth{\@biblabel{#1}}%
            \leftmargin\labelwidth
            \advance\leftmargin\labelsep
            \@openbib@code
            \usecounter{enumiv}%
            \let\p@enumiv\@empty
            \renewcommand\theenumiv{\@arabic\c@enumiv}}%
      \sloppy
      \clubpenalty4000
      \@clubpenalty \clubpenalty
      \widowpenalty4000%
      \sfcode`\.\@m}
     {\def\@noitemerr
       {\@latex@warning{Empty `thebibliography' environment}}%
      \endlist}
\makeatother



%\usepackage{etoolbox}
%\makeatletter
%\patchcmd{\thebibliography}{%
%  \chapter*{\bibname}\@mkboth{\MakeUppercase\bibname}{\MakeUppercase\bibname}}{%
%  \section{مراجع}}{}{}
%\makeatother



%%%%%%%%%%%%%%%%%%%%%%%%%%%%%%

\newcommand{\hex}[1]{\text{\lr{#1}}}
\newcommand{\textttc}[1]{\textcolor{blue}{\lr{\texttt{#1}}}}
\newcommand{\textttt}[1]{\lr{\texttt{#1}}}
\newcommand{\cinline}[1]{\colorbox{black!20!white}{\lr{\lstinline{#1}}}}











%%%%%%%%%%%%%%%%%%%%%%%%

\begin{document}
% Title Page
\pagestyle{empty}
%%%%%%%%%%%%%%%%%%%%%%%%
\begin{titlepage}
\begin{center}
\includegraphics[width=3cm]{logo.png} \\
{\fontsize{12}{12}\selectfont \bfseries دانشگاه صنعتی اصفهان}\\
{\fontsize{12}{12}\selectfont \bfseries دانشکده مهندسی برق و کامپیوتر}\\
\vspace{2cm}
{\fontsize{14}{14}\selectfont \bfseries پروژه درس مهندسی نرم‌افزار} \\[5mm]
{\fontsize{18}{18}\selectfont \bfseries گروه نرمینو} \\[5mm]
{\textcolor{Blue}{\rule{\textwidth}{1mm}}}  \\
{\fontsize{24}{24}\selectfont \bfseries
سیستم رأی‌گیری الکترونیکی امن
}\\[5mm]
{\fontsize{16}{16}\selectfont \bfseries نیازمندی‌ها}
\\
{\textcolor{Blue}{\rule{\textwidth}{1mm}}}  \\
\vspace{1cm}

{\fontsize{14}{14}\selectfont \bfseries اعضای گروه:} \\[3mm]
{\fontsize{16}{16}\selectfont \bfseries نگین دشتی } \\
{\fontsize{16}{16}\selectfont \bfseries محمدرضا ماجد} \\
{\fontsize{16}{16}\selectfont \bfseries سید رضا موسوی } \\
{\fontsize{16}{16}\selectfont \bfseries محمدصالح ناصح  } \\
\vspace{1cm}
{\fontsize{14}{14}\selectfont \bfseries استاد راهنما: } \\[3mm]
{\fontsize{16}{16}\selectfont \bfseries امیرارسلان یاوری } \\
\vfill
{\fontsize{16}{16}\selectfont \bfseries\today}
\end{center}
\end{titlepage}

%%%%%%%%%%%%%%%%%%%%%%%%
\clearpage
\tableofcontents
\clearpage
%%%%%%%%%%%%%%%%%%%%%%%%
\pagestyle{fancy}
\lhead{}
\rhead{}
\chead{نیازمندی‌های پروژه سیستم رأی‌گیری الکترونیکی امن - گروه نرمینو - دانشگاه صنعتی اصفهان}
%%%%%%%%%%%%%%%%%%%%%%%%



%%%%%%%%%%%%%%%%%%%%%%%%
\section{نیازمندی‌های کاربردی}
%%%%%%%%%%%%%%%%%%%%%%%%

\subsection{نیازمندی‌های کاربر رأی‌دهنده}
\begin{itemize}[leftmargin=*]
\item
ساده بودن و کاربرپسند بودن برنامه رأی‌گیری و مقرون به صرفه بودن آن برای کاربر
\item
ثبت‌نام و احراز هویت امن 
(\lr{register})
برای ورود به سیستم رأی‌گیری
(\lr{login})
\item
ثبت رأی به صورت امن و غیرقابل تغییر و با حفظ حریم خصوصی کاربر
\item
قابلیت اثبات فردی و انتها به انتهای رأی داده شده بدون افشای آن (یا با دانش صفر)
\item
ارتباط امن با شبکه بلاک‌چین و ارسال تراکنش بدون افشای هویت واقعی
\item
قابلیت ارسال سؤال برای بخش پشتیبانی و دریافت جواب در کوتاه‌ترین زمان ممکن
\item
قابلیت دیدن راهنمایی‌های کامل در مورد روش کار با برنامه رأی‌گیری
\item
دریافت اعلان‌های مرتبط با هر یک از مراحل رأی‌گیری از طریق پیامک و برنامه انتخابات
\item
قابلیت ناشناس بودن رأی‌گیری و عدم افشای رأی کاربر در هر یک از مراحل
\item
قابلیت راستی‌آزمایی رأی‌گیری توسط کاربر در پایان رأی‌گیری
\item
قابلیت دیدن گزارش‌ها و نتایج رأی‌گیری و تأیید عمومی و انتها به انتهای آن
\item
قابلیت شفاف بودن انتخابات در تمامی مراحل و متمرکز نبودن آن
\item
دسترسی به نتایج انتخابات به صورت شفاف و مطمئن
\end{itemize}



\subsection{نیازمندی‌های مدیران سیستم و مسئولان برگزاری انتخابات}
\begin{itemize}[leftmargin=*]
\item
قابلیت استفاده از سیستم رأی‌گیری برای همه افراد با حداقل امکانات موجود 
\item
ایجاد یک فرآیند انتخابات به صورت امن و بدون امکان دستکاری عامل بیرونی
\item
چک کردن احراز هویت و واجد شرایط بودن رأی‌دهندگان به یک روش امن و مطمئن
\item
قابلیت جلوگیری از تقلب در انتخابات و چند بار رأی دادن افراد با هزینه کم
\item
قابل تأیید بودن فردی، عمومی و انتها به انتهای رأی‌گیری
\item
مقرون به صرفه بودن برگزاری انتخابات نسبت به روش‌های سنتی
\item
نظارت و حسابرسی شفاف آرا در تمامی مراحل رأی‌گیری
\item
محرمانه بودن رأی‌گیری 
\item
یکپارچگی سیستم رأی‌گیری و مقاوم بودن در برابر حملات و آسیب‌پذیری‌های امنیتی
\item
قابل اطمینان بودن سیستم از اینکه هیچ رأیی حتی در صورت مشکلات سیستم و خرابی برخی از سرورها از بین نمی‌رود. 
\item
مدیریت و تنظمیات انتخابات به صورت امن و مطمئن
\end{itemize}






%%%%%%%%%%%%%%%%%%%%%%%%
\section{نیازمندی‌های غیرکاربردی}
%%%%%%%%%%%%%%%%%%%%%%%%
\subsection{نیازمندی‌های کاربر رأی‌دهنده}
\begin{itemize}[leftmargin=*]
\item
ارتباط امن و رمزنگاری شده انتها به انتها با مدیران سیستم در طی مراحل انتخابات
\item
قابلیت تولید کلید عمومی و خصوصی به یک روش امن برای ناشناس ماندن در فرآیند انتخابات
\item
قابلیت امضای قرارداد هوشمند و تولید 
\lr{ZKSMP}
برای اثبات رأی‌دهی با دانش صفر
\item
قابلیت ارتباط امن و رمزنگاری شده انتها به انتها با شبکه بلاک‌چین و ارسال تراکنش (برگه رأی یا توکن)
\end{itemize}



\subsection{نیازمندی‌های مدیران سیستم و مسئولان برگزاری انتخابات}
\begin{itemize}[leftmargin=*]
\item
قابلیت استفاده از برنامه در همه سیستم‌عامل‌ها و مرورگرهای مدرن
\item
رمزنگاری داده‌ها و محافظت از اطلاعات هویتی کاربران
\item
ارتباط امن و رمزنگاری شده انتها به انتها با هر یک از اجزای فرآیند رأی‌دهی مثل کاربران، سامانه ثبت احوال و غیره 
\item
چک کردن احراز هویت و واجد شرایط بودن رأی‌دهندگان به یک روش امن و بدون امکان تقلب
\item
تشخیص ربات نبودن کاربر با استفاده از روش‌های مدرن و بروز مثل روش‌های هوش مصنوعی
\item
قابل تأیید بودن فردی، عمومی و انتها به انتهای رأی‌گیری
\item
مقرون به صرفه بودن برگزاری انتخابات نسبت به روش‌های سنتی
\item
نظارت و حسابرسی شفاف آرا در تمامی مراحل رأی‌گیری
\item
محرمانه بودن رأی‌گیری 
\item
یکپارچگی سیستم رأی‌گیری و مقاوم بودن در برابر حملات و آسیب‌پذیری‌های امنیتی
\item
قابل اطمینان بودن سیستم از اینکه هیچ رأیی حتی در صورت مشکلات سیستم و خرابی برخی از سرورها از بین نمی‌رود. 
\item
مدیریت و تنظمیات انتخابات به صورت امن و مطمئن
\item
مقیاس پذیری سیستم برای مدیریت حجم بالای کاربران و تراکنش‌ها
\item
استفاده از بلاک‌چین برای ثبت و ذخیره آرا
\item
پشتیبان‌گیری منظم از داده‌ها و تراکنش‌ها
\item
پیاده‌سازی 
\lr{CI/CD} 
برای توسعه سریع‌تر و پایدار
\item
نظارت مداوم بر عملکرد سیستم 
(\lr{Monitoring})
\end{itemize}






\end{document}    
