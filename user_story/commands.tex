% !TEX TS-program = xelatex
% !TeX root=Narmino_SE_Project_Proposal.tex


\usepackage[utf8]{inputenc}
\usepackage{amsmath,amsfonts,amssymb,amsthm}
\usepackage{mathtools}

\usepackage{geometry}
\geometry{a4paper,hmargin=20mm,vmargin=25mm,head=3ex,foot=8ex}
\usepackage{titlesec} %[newparttoc]
\usepackage{fancyhdr}
\usepackage{graphicx}
\usepackage[usenames,dvipsnames]{color,xcolor}
\usepackage{listings}
\usepackage{fourier}
\usepackage[most]{tcolorbox}
\usepackage{xcolor}
\usepackage[colorlinks=true, linkcolor=Blue,urlcolor=Blue,citecolor=Blue, pagebackref, linkbordercolor=Blue]{hyperref} %%%
\usepackage{xurl}
\usepackage{enumitem}
\usepackage{float}
%\usepackage{subcaption}
\usepackage[font=small,skip=0pt]{caption}
\usepackage{subfigure}
%%%%%%%%%%%%%%%%%%%%%%%
\usepackage{multirow,array}

\usepackage{algorithm}
\usepackage{algorithmic}
\usepackage{realboxes}

\usepackage{tikz}
\usetikzlibrary{mindmap}

\usepackage{xpatch}
\xpretocmd\headrule{\color{Blue}}{}{\PatchFailed}
%\newlength{\myheight}
%\lfoot{}
%\settoheight{\myheight}{\thepage}
%\cfoot{\color{Blue}\thepage}
\renewcommand\headrulewidth{3pt}
%\renewcommand\footrulewidth{0pt}



%%%%%%%%%%%%%%%%%%%%%%%%%% TOC

\usepackage{titletoc}

%%%%%%%%%%%%%%%%%%%%%%%
\usepackage[extrafootnotefeatures]{xepersian} %
\settextfont{HM FMitra}   %{HM FMitra}
\setdigitfont{HM FMitra}   %{HM FMitra}
\setlatintextfont{Times New Roman} %[Scale=.9]
\defpersianfont\titrfont{HM FMitra}   %{HM FMitra}
%\paragraphfootnotes
\usepackage{perpage}
\MakePerPage{footnote}
\linespread{1.6}



\DeclarePairedDelimiter{\abs}{\lvert}{\rvert}
%%%%%%%%%%%%%%%%%%
\setlength{\footmarkwidth}{0pt}

\setlength{\hsize}{\textwidth}
\paragraphfootnotes
%\makeatletter
%\newif\ifLTRfootnoterule
%\let\@footnotetext\m@mold@footnotetext
%\gdef\mem@makecol{%
%  \m@m@makecolintro
%  \ifvoid\footins
%  \ifvoid\footinsv@r\else \LTRfootnoteruletrue\fi
%    \setbox\@outputbox \box\@cclv
%  \else
%    \m@mopfootnote
%  \fi
%  \m@mdoextrafeet
%  \m@m@makecolfloats
%  %\m@mopsidebar
%  \m@m@makecoltext
%  \global \maxdepth \@maxdepth
%  }
%\renewcommand*{\@footstartv@r}{%
%    \vskip 3mm %\bigskipamount
%    \leftskip=\z@
%    \rightskip=\z@
%    \ifLTRfootnoterule\footnoterule\fi}
%\makeatother
%%%%%%%%%%%%%%
%%%%%%%%%%%%%%%%%
%\setcounter{secnumdepth}{4}
%\setcounter{tocdepth}{3}
\newtcolorbox{SAMBox}[3]{colback=#1, colframe=#2, fonttitle=\bfseries,title={\LTR #3},breakable,enhanced}
\newtcolorbox{SAMBoxP}[3]{colback=#1, colframe=#2, fonttitle=\bfseries,title={#3},breakable,enhanced}
%%%%%%%%%%%%%%%%%


%\lstset{ 
%	language=Python,
%  backgroundcolor=\color{red!10},   % choose the background color; you must add \usepackage{color} or \usepackage{xcolor}; should come as last argument
%  basicstyle=\fontsize{10}{10}\selectfont \tt,        % the size of the fonts that are used for the code
%  breakatwhitespace=false,         % sets if automatic breaks should only happen at whitespace
%  breaklines=true,                 % sets automatic line breaking
%  captionpos=b,                    % sets the caption-position to bottom
%  commentstyle=\color{purple!40!black},    % comment style
%  deletekeywords={...},            % if you want to delete keywords from the given language
%  escapeinside={\%*}{*)},          % if you want to add LaTeX within your code
%  extendedchars=true,              % lets you use non-ASCII characters; for 8-bits encodings only, does not work with UTF-8
%  frame=single,	                   % adds a frame around the code   %  frame=L,
%  keepspaces=true,                 % keeps spaces in text, useful for keeping indentation of code (possibly needs columns=flexible)
%  keywordstyle=\color{blue},       % keyword style
% % language=Octave,                 % the language of the code
%  morekeywords={*,...},            % if you want to add more keywords to the set
%  numbers=left,                    % where to put the line-numbers; possible values are (none, left, right)
%  numbersep=5pt,                   % how far the line-numbers are from the code
%  numberstyle=\tiny\color{green!40!black}, % the style that is used for the line-numbers
%  rulecolor=\color{black},         % if not set, the frame-color may be changed on line-breaks within not-black text (e.g. comments (green here))
%  showspaces=false,                % show spaces everywhere adding particular underscores; it overrides 'showstringspaces'
%  showstringspaces=false,          % underline spaces within strings only
%%  showtabs=false,                  % show tabs within strings adding particular underscores
%  stepnumber=1,                    % the step between two line-numbers. If it's 1, each line will be numbered
%  stringstyle=\color{orange},     % string literal style
%%  tabsize=2,	                   % sets default tabsize to 2 spaces
%%  title=\lstname,                   % show the filename of files included with \lstinputlisting; also try caption instead of title
%%  belowcaptionskip=\baselineskip,
%  xleftmargin=\parindent,
%}



\lstset{ 
	language=C++,
  	backgroundcolor=\color{red!10},  
  	basicstyle=\fontsize{10}{10}\selectfont \tt,    
 	 breakatwhitespace=false,  
  	breaklines=true,  
 	 captionpos=b,  
	  commentstyle=\color{purple!40!black},    
	  deletekeywords={...},          
	  escapeinside={\%*}{*)},  
	  extendedchars=true,  
 	 frame=single,	
 	 keepspaces=true,    
 	 keywordstyle=\color{blue},   
 	 morekeywords={*,...},        
 	 numbers=left, 
  	numbersep=5pt,     
 	 numberstyle=\tiny\color{green!40!black}, 
 	 rulecolor=\color{black},  
 	 showspaces=false, 
 	 showstringspaces=false,   
	%  showtabs=false,         
 	stepnumber=1,  
	 stringstyle=\color{orange},    
	%  tabsize=2,	 
	%  title=\lstname,  
	  %belowcaptionskip=\baselineskip,
 	 xleftmargin=0pt%\parindent,
}


\lstdefinestyle{DOS}
{
    backgroundcolor=\color{black},
    basicstyle=\scriptsize\color{white}\ttfamily
}




\newcommand{\code}[1]{\lr{\lstinline{#1}}}

%%%%%%%%%%%%%%%%%

\definecolor{Blue}{rgb}{0,0,.55}
\definecolor{Blue2}{rgb}{.55,0,.55}


\titleformat{\section}
{\color{Blue} \Large\bfseries}
{\thesection}{1em}
{}[\titlerule]

\titleformat{\subsection}
{\color{Blue} \large\bfseries}
{\thesubsection}
{1em}{}
\titleformat{\subsubsection}
{\color{Blue} \bfseries}
{\thesubsubsection}
{1em}{}



%%TITLE TOC
\titlecontents{part}[0pt]{\color{Blue}\large\bfseries\protect\addvspace{15pt}}%
{}{\color{Blue}\partname{} }%
{\color{Blue}\enspace\hfill\contentspage\enspace}%
[\titlerule]



%%%%%%%%%%%%%%%%%%%%%

\newtheorem{theorem}{قضیه}[section]
\newtheorem{lemma}[theorem]{لم}
\newtheorem{proposition}[theorem]{گزاره}
\theoremstyle{definition}
\newtheorem{definition}[theorem]{تعریف}
\newtheorem{example}[theorem]{مثال}
\newtheorem{remark}[theorem]{ملاحظه}
\newtheorem{obs}[theorem]{مشاهده}
\newtheorem{rull}[theorem]{قانون}
\newtheorem*{namad}{نمادگذاری}

%%%%%%%%%%%%%%%%%%%%

\graphicspath{{Figs/}}

%%%%%%%%%%%%%%%%%%%%%%%%
%\renewcommand\bibname{مراجع}
\makeatletter
\renewenvironment{thebibliography}[1]
     {\section{مراجع}% <-- this line was changed from \chapter* to \section*
      \@mkboth{\MakeUppercase\bibname}{\MakeUppercase\bibname}%
      \list{\@biblabel{\@arabic\c@enumiv}}%
           {\settowidth\labelwidth{\@biblabel{#1}}%
            \leftmargin\labelwidth
            \advance\leftmargin\labelsep
            \@openbib@code
            \usecounter{enumiv}%
            \let\p@enumiv\@empty
            \renewcommand\theenumiv{\@arabic\c@enumiv}}%
      \sloppy
      \clubpenalty4000
      \@clubpenalty \clubpenalty
      \widowpenalty4000%
      \sfcode`\.\@m}
     {\def\@noitemerr
       {\@latex@warning{Empty `thebibliography' environment}}%
      \endlist}
\makeatother



%\usepackage{etoolbox}
%\makeatletter
%\patchcmd{\thebibliography}{%
%  \chapter*{\bibname}\@mkboth{\MakeUppercase\bibname}{\MakeUppercase\bibname}}{%
%  \section{مراجع}}{}{}
%\makeatother



%%%%%%%%%%%%%%%%%%%%%%%%%%%%%%

\newcommand{\hex}[1]{\text{\lr{#1}}}
\newcommand{\textttc}[1]{\textcolor{blue}{\lr{\texttt{#1}}}}
\newcommand{\textttt}[1]{\lr{\texttt{#1}}}
\newcommand{\cinline}[1]{\colorbox{black!20!white}{\lr{\lstinline{#1}}}}









